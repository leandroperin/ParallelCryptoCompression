\chapter{Conclusão}
\label{cap6}

O algoritmo \gmpr funciona corretamente, cumprindo seu objetivo, porém apresenta um desempenho insatisfatório. Este trabalho busca analisar o código e melhorá-lo, garantindo que seu tempo de execução diminua e o torne possível de ser utilizado em alguma aplicação real.

A utilização das tecnologias \openMP, \mpi e/ou \cuda/\opencl aparentam ser as melhores opções disponíveis para melhorar o tempo de execução do algoritmo de cripto-compressão \gmpr. Elas serão implementadas e terão os resultados medidos, como forma de garantir os ganhos de desempenho.

Ter o algoritmo funcionando de forma eficiente é uma conquista muito interessante, pois o mesmo pode ser aplicado em serviços \textit{web} e na nuvem, reduzindo o armazenamento utilizado, os riscos de segurança e também o tempo necessário para enviar informações através da Internet.

O uso deste trabalho em projetos futuros é bastante possível, afinal esse algoritmo pode ser usado em qualquer projeto que tenha restrições de segurança das informações ou restrições de armazenamento. Há também a possibilidade desse projeto ser estendido para outras áreas, como a cripto-compressão de um banco de dados, dentre outras.